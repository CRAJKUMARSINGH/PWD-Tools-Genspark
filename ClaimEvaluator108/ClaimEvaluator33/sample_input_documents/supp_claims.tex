\documentclass[12pt,letterpaper]{article}
\usepackage[margin=1in]{geometry}
\usepackage{enumitem}
\usepackage{amsmath,amssymb}
\usepackage{parskip}
\usepackage{microtype}
\usepackage{times}

\begin{document}
	
	% Title
	\begin{center}
		\textbf{Supplementary Claim Heads for Contract Package ARTL/WIPL/2021/001} \\
		Period: September 2023--June 2025 (22 months) \\
		Submission Date: September 26, 2025
	\end{center}
	
	% Overview
	\section*{Overview}
	These supplementary claim heads (8--14) address additional costs and losses beyond the primary prolongation claims (A--I, \texttt{INR 451.55 Crores}, Annexure 1). Each head is tied to Employer-attributable delays, primarily failure to provide hindrance-free Right of Way (RoW) per Clause 8.1 of the Construction Contract and Clause 2.1 of the Supplementary Agreement, impacting the critical path (Appendix X -- Delay Analysis). Narratives detail computation methods, contractual/legal basis, and assumptions for clarity, ensuring alignment with actual deployment in September 2023 (Annexure 1.1.1). Claims 15 and 16 are merged with existing heads H and F to avoid overlap. All claims are supported by contemporaneous records (MPRs, notices, invoices) to mitigate global claim risks (\textit{Associate Builders v. DDA}, 2015).
	
	% Claim 8: Disruption Costs
	\section*{8. Disruption Costs}
	\textbf{Amount}: \texttt{INR 62.00 Crores} (Appendix H)
	
	In addition to time-related prolongation, the Contractor has suffered substantial disruption losses arising from fragmented and stop--start execution of the Works. These losses are distinct from prolongation, as they relate to reduced productivity and inflated unit costs on the work fronts that were made partially available. Despite mobilization in accordance with the Contract, execution was repeatedly hindered due to piecemeal site handovers, delays in removal of utilities and trees, and late approvals of design and change of scope works. As a result, resources deployed at site were often idle or forced to work under suboptimal sequencing, leading to measurable productivity losses.
	
	\subsection*{Computation Basics}
	\begin{itemize}
		\item Planned productivity: 500 cu.m/day of earthwork (Apr--Aug 2023, Schedule 9).
		\item Actual productivity: 300 cu.m/day over 660 days (Sep 2023--Jun 2025).
		\item Affected quantum: \texttt{INR 180 Crores} (BOQ items 2.01--2.03, earthwork and subgrade).
		\item Productivity loss: 40\%, mitigated to 35\% via re-sequencing.
		\item Cost calculation:
		\[
		\texttt{INR 180 Crores} \times 0.35 = \texttt{INR 63 Crores}
		\]
		\item Adjusted claim: \texttt{INR 62 Crores} for conservatism.
		\item Supporting records: MPRs, productivity logs, notices (Sep 2023, Appendix H).
	\end{itemize}
	
	\subsection*{Basis}
	\begin{itemize}
		\item \textbf{Contractual}: Clause 2.1 (hindrance-free RoW), Clause 13.1 (disruption compensation).
		\item \textbf{Factual}: Piecemeal RoW, utility delays, late approvals (notices Feb 2022, Sep 2023) reduced efficiency (MPRs, Appendix H).
		\item \textbf{Legal}: Section 73, Indian Contract Act, 1872; \textit{NHAI v. HCC} (2015, Delhi HC) awarded 25\% productivity loss; SCL Delay and Disruption Protocol (2017) endorses measured mile methodology.
		\item The Contractor reserves the right to update this claim with additional productivity data.
	\end{itemize}
	
	\subsection*{Assumptions}
	\begin{itemize}
		\item 660-day period aligns with prolongation; adjusted from 274 days.
		\item \texttt{INR 180 Crores} based on BOQ items (Appendix H); assumes earthwork as primary scope.
		\item 35\% loss assumes mitigation (MPRs, Annexure 1.1.1: 182 skilled, 273 unskilled, 12 excavators, Sep 2023).
		\item Excludes labour/equipment in Claims A, B to avoid overlap.
	\end{itemize}
	
	\subsection*{Narrative}
	Fragmented RoW and utility delays disrupted earthwork execution, reducing productivity from 500 to 300 cu.m/day. The measured mile analysis, supported by MPRs and logs, quantifies a 35\% efficiency loss on \texttt{INR 180 Crores} of work. Costs are distinct from prolongation (Claims A, B), tied to Clause 2.1 breaches, and supported by \textit{NHAI v. HCC}. Appendix H includes productivity logs, notices, and BOQ details, ensuring transparency.
	
	% Claim 9: Acceleration Costs
	\section*{9. Acceleration Costs}
	\textbf{Amount}: \texttt{INR 25.00 Crores} (Appendix I)
	
	The Contractor incurred acceleration costs to mitigate Employer-caused delays and maintain progress despite delayed or denied Extension of Time (EOT) approvals. Unlike prolongation or disruption, this head covers premiums paid to accelerate performance to avoid liquidated damages and meet milestones. Despite excusable delays, timely EOT approvals were delayed, forcing acceleration measures including overtime, additional plant, and re-sequencing.
	
	\begin{itemize}
		\item Overtime and double shifts for critical activities.
		\item Additional plant, equipment, and skilled personnel beyond original planning.
		\item Re-sequencing works with compressed durations.
	\end{itemize}
	
	\subsection*{Computation Basics}
	\begin{itemize}
		\item 3-month acceleration period:
		\begin{itemize}
			\item Overtime/extra shifts: \texttt{INR 0.9 Crores}.
			\item Additional plant: \texttt{INR 3.0 Crores}.
			\item Total: \texttt{INR 3.9 Crores}.
		\end{itemize}
		\item Total costs:
		\begin{itemize}
			\item Overtime: \texttt{INR 0.9 Crores/month} $\times$ 22 months = \texttt{INR 19.8 Crores} (Annexure 1.1.1: 182 skilled, 273 unskilled).
			\item Plant: 5 excavators $\times$ \texttt{INR 2.88 Lakhs/month} $\times$ 22 months = \texttt{INR 3.17 Crores} (Annexure 1.5.1).
			\item Supervision/ancillary: \texttt{INR 2.03 Crores}.
		\end{itemize}
		\item Total claim:
		\[
		\texttt{INR 19.8 Crores} + \texttt{INR 3.17 Crores} + \texttt{INR 2.03 Crores} = \texttt{INR 25 Crores}
		\]
		\item Supporting evidence: EOT notices, resource logs (Appendix I).
	\end{itemize}
	
	\subsection*{Basis}
	\begin{itemize}
		\item \textbf{Contractual}: Clause 8.6 (acceleration), Clause 10.3 (EOT provisions).
		\item \textbf{Factual}: Delayed EOT approvals (Oct 2023 notices) forced acceleration to avoid liquidated damages (Clause 8.7, MPRs).
		\item \textbf{Legal}: Section 55, Indian Contract Act, 1872; \textit{Oil India Ltd v. Essar Oil Ltd} (2019, Delhi HC) recognized constructive acceleration.
		\item The Contractor reserves entitlement to full acceleration costs.
	\end{itemize}
	
	\subsection*{Assumptions}
	\begin{itemize}
		\item Overtime applied to labour (Annexure 1.1.1); plant aligns with Annexure 1.5.1.
		\item Assumes 22-month impact per EOT delays.
		\item Ancillary costs (\texttt{INR 2.03 Crores}) per industry norms.
	\end{itemize}
	
	\subsection*{Narrative}
	To mitigate Employer-caused delays, WIPL accelerated critical activities via overtime and additional plant, incurring costs beyond the Contract Price. Supported by EOT notices and resource logs (Appendix I), the claim is tied to Clause 8.6 breaches and avoids overlap with prolongation (Claims A, B). \textit{Oil India v. Essar} supports recovery.
	
	% Claim 10: Extended Bond, Insurance, and Guarantee Costs
	\section*{10. Extended Bond, Insurance, and Guarantee Costs}
	\textbf{Amount}: \texttt{INR 9.15 Crores} (Appendix J)
	
	The prolonged project duration necessitated extending performance bonds, advance payment guarantees, and project-wide insurance beyond the original period. Equipment insurance is included in Claim A (Annexure 1.5), but broader project securities incurred additional costs.
	
	\begin{itemize}
		\item Performance bonds and guarantees (Clause 4.2) extended for 22 months.
		\item Advance payment guarantees renewed beyond original expiry.
		\item Project insurance (Contract Works, Third-Party Liability, Workmen’s Compensation) extended.
	\end{itemize}
	
	\subsection*{Computation Basics}
	\begin{itemize}
		\item Annual premium: \texttt{INR 5 Crores}.
		\item Extension: 1.83 years (Sep 2023--Jun 2025).
		\item Total:
		\[
		\texttt{INR 5 Crores/year} \times 1.83 = \texttt{INR 9.15 Crores}
		\]
		\item Supporting records: Invoices, premium schedules (Appendix J).
	\end{itemize}
	
	\subsection*{Basis}
	\begin{itemize}
		\item \textbf{Contractual}: Clause 4.2 (performance securities), Clause 8.4 (prolongation costs).
		\item \textbf{Factual}: RoW delays (Clause 8.1) required renewals (bank/insurer records, Appendix J).
		\item \textbf{Legal}: Section 73, Indian Contract Act, 1872; \textit{NHAI v. ITD Cementation} (2015, Supreme Court) recognized prolongation-related costs.
	\end{itemize}
	
	\subsection*{Assumptions}
	\begin{itemize}
		\item Premiums exclude equipment insurance (Claim A).
		\item \texttt{INR 5 Crores/year} based on 1--2\% of \texttt{INR 864.71 Crores} project value.
		\item 22-month extension aligns with prolongation.
	\end{itemize}
	
	\subsection*{Narrative}
	RoW delays extended performance bonds, guarantees, and project insurance, incurring additional premiums. Costs are verified by invoices (Appendix J), distinct from Claim A, and recoverable under Clause 4.2. \textit{NHAI v. ITD Cementation} supports entitlement.
	
	% Claim 11: Escalation Beyond Contract Provisions
	\section*{11. Escalation Beyond Contract Provisions}
	\textbf{Amount}: \texttt{INR 15.00 Crores} (Merged with Claim D, Total \texttt{INR 46.23 Crores}, Annexure 1.8)
	
	Covers Building Stone/Rough Stone/Road Metal, Morrum/Gravel Ordinary Earth. Additional material and labour escalation pending feasibility.
	
	\subsection*{Computation Basics}
	\begin{itemize}
		\item Cement:
		\[
		(\texttt{INR 7,500/MT} - \texttt{INR 6,000/MT}) \times 50,000 \, \text{MT} = \texttt{INR 7.50 Crores}
		\]
		\item Steel:
		\[
		(\texttt{INR 75,000/MT} - \texttt{INR 60,000/MT}) \times 50,000 \, \text{MT} = \texttt{INR 7.50 Crores}
		\]
		\item Total: \texttt{INR 15.00 Crores}.
		\item Merged with Claim D: \texttt{INR 31.23 Crores} (aggregates/earth) + \texttt{INR 15.00 Crores} = \texttt{INR 46.23 Crores} (Appendix K).
	\end{itemize}
	
	\subsection*{Basis}
	\begin{itemize}
		\item \textbf{Contractual}: Clause 14.1 (price adjustment, if applicable), Clause 13.1 (delay-induced costs).
		\item \textbf{Factual}: Market price rises for cement/steel (2023--25, supplier quotes, WPI indices, Appendix K).
		\item \textbf{Legal}: Section 73, Indian Contract Act, 1872; \textit{NHAI v. Som Datt Builders} (2009, Supreme Court) supported escalation claims.
		\item Merged with Claim D to streamline, pending Clause 14.1 confirmation.
	\end{itemize}
	
	\subsection*{Assumptions}
	\begin{itemize}
		\item Quantities (50,000 MT each) from BOQ logs (Annexure 1.8).
		\item Rates (\texttt{INR 7,500/MT} cement) from 2023--25 market data.
		\item No labour escalation claimed, pending contract provisions.
	\end{itemize}
	
	\subsection*{Narrative}
	Prolonged execution due to RoW delays increased cement and steel costs beyond aggregates/earth in Claim D. Quantified using BOQ quantities and market rates, the claim is supported by invoices (Appendix K) and tied to Clause 13.1. Merged with Claim D for clarity, per \textit{NHAI v. Som Datt Builders}.
	
	% Claim 12: Environmental and Regulatory Compliance Costs
	\section*{12. Environmental and Regulatory Compliance Costs}
	\textbf{Amount}: \texttt{INR 11.00 Crores} (Appendix L)
	
	Prolongation extended obligations for environmental and regulatory compliance, unforeseen in Contract pricing. Delays extended borrow areas, quarries, and material extraction sites, requiring renewed clearances. Additional costs covered monitoring, dust suppression, and MoEFCC reporting. Fines excluded due to insufficient evidence.
	
	\subsection*{Computation Basics}
	\begin{itemize}
		\item Monthly cost (monitoring, dust control, liaison): \texttt{INR 0.5 Crores}.
		\item Duration: 22 months (Sep 2023--Jun 2025).
		\item Total:
		\[
		\texttt{INR 0.5 Crores/month} \times 22 = \texttt{INR 11 Crores}
		\]
		\item Supporting records: Invoices, MoEFCC communications (Appendix L).
	\end{itemize}
	
	\subsection*{Basis}
	\begin{itemize}
		\item \textbf{Contractual}: Clause 4.18 (compliance with laws), Clause 8.4 (prolongation costs).
		\item \textbf{Factual}: Extended quarry operations due to RoW delays (MoEFCC records, Appendix L).
		\item \textbf{Legal}: Environment Protection Act, 1986; EIA Notification, 1994; Section 73, Indian Contract Act, 1872; MoEFCC cases (2023) recognized compliance costs.
	\end{itemize}
	
	\subsection*{Assumptions}
	\begin{itemize}
		\item \texttt{INR 0.5 Crores/month} based on highway project norms.
		\item Fines excluded, pending MoEFCC notices.
		\item Costs tied to 22-month prolongation (Annexure 1.1.1).
	\end{itemize}
	
	\subsection*{Narrative}
	RoW delays extended environmental compliance obligations, incurring costs for monitoring, dust control, and quarry renewals. Quantified conservatively at \texttt{INR 11 Crores}, the claim is supported by MoEFCC correspondence and invoices (Appendix L), tied to Clause 8.4, and aligned with 2023 precedents.
	
	% Claim 13: Idle Time and Standby Costs for Subcontractors
	\section*{13. Idle Time and Standby Costs for Subcontractors}
	\textbf{Amount}: \texttt{INR 20.00 Crores} (Appendix M)
	
	Prolonged delays caused idle time and standby costs for subcontractors, recoverable under back-to-back arrangements. These costs, incurred by subcontractors, are passed through to the Employer.
	
	\begin{itemize}
		\item Subcontractors mobilized resources per approved schedules.
		\item Partial site handovers, utility delays, and scope revisions caused idle periods.
		\item Compensation paid to maintain subcontractor availability.
	\end{itemize}
	
	\subsection*{Computation Basics}
	\begin{itemize}
		\item Standby rate: \texttt{INR 0.20 Crores/day}.
		\item Duration: 100 days (adjusted from 200, per MPRs).
		\item Total:
		\[
		\texttt{INR 0.20 Crores/day} \times 100 = \texttt{INR 20 Crores}
		\]
		\item Supporting records: Subcontractor notices, certified statements, invoices (Appendix M).
	\end{itemize}
	
	\subsection*{Basis}
	\begin{itemize}
		\item \textbf{Contractual}: Clause 13.1 (delay-induced costs), back-to-back subcontract clauses.
		\item \textbf{Factual}: Partial RoW and utility delays caused idle periods (subcontractor notices, MPRs, Appendix M).
		\item \textbf{Legal}: Indian Arbitration and Conciliation Act, 1996; \textit{Prasar Bharati v. Multi Channel} (2019, Delhi HC) upheld pass-through claims.
	\end{itemize}
	
	\subsection*{Assumptions}
	\begin{itemize}
		\item 100 days based on MPRs, Annexure 1.1.1 (e.g., 12 excavators idle).
		\item Rates from subcontractor agreements; excludes Claims A, B.
		\item Costs distinct from prolongation.
	\end{itemize}
	
	\subsection*{Narrative}
	Subcontractors incurred standby costs due to fragmented RoW and utility delays, passed through to WIPL under back-to-back agreements. Quantified at \texttt{INR 20 Crores} for 100 days, the claim is supported by notices and certified statements (Appendix M), tied to Clause 13.1, and upheld by \textit{Prasar Bharati}.
	
	% Claim 14: Variations and Change of Scope
	\section*{14. Claims for Variations and Change of Scope as Prolongation Drivers}
	\textbf{Amount}: \texttt{INR 35.00 Crores} (Appendix N)
	
	Employer-instructed variations and delayed approvals of scope changes, including major structures, extended the project beyond original assumptions, contributing to prolongation.
	
	\begin{itemize}
		\item Major structures introduced, not foreseen at tender.
		\item Piecemeal scope changes required redesign and re-mobilization.
		\item Delayed variation approvals caused critical path slippage.
	\end{itemize}
	
	\subsection*{Computation Basics}
	\begin{itemize}
		\item Direct costs (additional works): \texttt{INR 30 Crores}.
		\item Prolongation costs: \texttt{INR 5 Crores}.
		\item Total:
		\[
		\texttt{INR 30 Crores} + \texttt{INR 5 Crores} = \texttt{INR 35 Crores}
		\]
		\item Supporting records: Computations, Engineer approvals, correspondence (Appendix N).
	\end{itemize}
	
	\subsection*{Basis}
	\begin{itemize}
		\item \textbf{Contractual}: Clause 13.1 (variations), Clause 8.4 (prolongation costs).
		\item \textbf{Factual}: New structures and delayed approvals extended critical path (Engineer instructions, delay notices, Appendix N).
		\item \textbf{Legal}: Section 73, Indian Contract Act, 1872; \textit{NHAI v. Som Datt Builders} (2009, Supreme Court) upheld variation-linked costs.
	\end{itemize}
	
	\subsection*{Assumptions}
	\begin{itemize}
		\item \texttt{INR 30 Crores} based on BOQ items (structures, 5.01--5.04).
		\item \texttt{INR 5 Crores} assumes 10\% prolongation impact (delay analysis).
		\item Supported by Annexure 1.1.1.
	\end{itemize}
	
	\subsection*{Narrative}
	Employer-instructed variations, including new structures, extended the project, incurring direct and time-related costs. Quantified at \texttt{INR 35 Crores} via certified BOQ items and delay analysis (Appendix N), the claim is recoverable under Clause 13.1, distinct from prolongation, per \textit{NHAI v. Som Datt Builders}.
	
	% Claim 15: Interest on Delayed Payments
	\section*{15. Interest on Delayed Payments}
	\textbf{Amount}: \texttt{INR 5.23 Crores} (Merged with Claim H, Annexure 1.9.3)
	
	The Contractor seeks interest on delayed payments, representing the time value of money withheld beyond contractual timelines, distinct from financing costs.
	
	\begin{itemize}
		\item Progress payments/certified amounts delayed.
		\item Advances repayable through execution delayed by Employer defaults.
		\item Interest calculated at commercial rates.
	\end{itemize}
	Illustrative example: \texttt{INR 100 Crores} outstanding for 365 days at 12\% p.a. = \texttt{INR 12 Crores}. Estimated entitlement: \texttt{INR 12--18 Crores}. Merged with Claim H (\texttt{INR 5.23 Crores}, RE Wall payments) to avoid overlap.
	
	\subsection*{Computation Basics}
	\begin{itemize}
		\item Held quantity: 80,602.63 sqm $\times$ rates (\texttt{INR 1,613.67/sqm} pre-Supplementary, \texttt{INR 1,679/sqm} post) = \texttt{INR 129.43 Crores}.
		\item Interest:
		\[
		\texttt{INR 129.43 Crores} \times 0.12 \times \frac{41}{12} = \texttt{INR 5.23 Crores}
		\]
		\item Supporting records: Casting/erection logs (Annexure 1.9.3), bank schedules (Appendix N).
	\end{itemize}
	
	\subsection*{Basis}
	\begin{itemize}
		\item \textbf{Contractual}: Clause 11.1 (payment terms), Clause 13.1.
		\item \textbf{Factual}: RoW delays withheld RE Wall payments (Annexure 1.9.3).
		\item \textbf{Legal}: Section 31(7), Arbitration and Conciliation Act, 1996; \textit{ONGC v. GT Beckfield} (2025, Supreme Court) clarified pendente lite interest.
	\end{itemize}
	
	\subsection*{Assumptions}
	\begin{itemize}
		\item Full quantity held due to RoW; 12\% rate standard.
		\item No additional interest to avoid double-counting.
	\end{itemize}
	
	\subsection*{Narrative}
	Interest on delayed RE Wall payments (Claim H) captures compensable delays, quantified at \texttt{INR 5.23 Crores} using verified quantities and logs (Annexure 1.9.3). Tied to Clause 11.1 breaches and supported by \textit{ONGC v. GT Beckfield}, the claim avoids overlap with Claim E.
	
	% Claim 16: Consequential Losses
	\section*{16. Consequential Losses}
	\textbf{Amount}: \texttt{INR 78.61 Crores} (Merged with Claim F, Annexure 1.9)
	
	Consequential losses from Employer-attributable delays include lost opportunities to bid for other projects due to resource tie-up, beyond direct prolongation costs.
	
	\begin{itemize}
		\item Plant, personnel, and financial capacity locked beyond 22-month schedule.
		\item Prevented pursuit of alternative projects, foreseeable at Contract execution.
		\item Extends to missed bids and reduced prequalification capacity; reputational losses excluded (Clause 13.8).
	\end{itemize}
	Hudson Formula estimates lost profits at \texttt{INR 10--20 Crores}. Merged with Claim F (\texttt{INR 78.61 Crores}) to avoid overlap.
	
	\subsection*{Computation Basics}
	\begin{itemize}
		\item Total work: \texttt{INR 864.71 Crores} $\div$ 22 months = \texttt{INR 39.31 Crores/month}.
		\item Profit (10\%):
		\[
		\left( \frac{\texttt{39.31}}{1.1} \right) \times 0.10 \times 22 = \texttt{INR 78.61 Crores}
		\]
		\item Supporting records: Bid records, declined tenders, financial statements (Appendix O).
	\end{itemize}
	
	\subsection*{Basis}
	\begin{itemize}
		\item \textbf{Contractual}: Clause 13.1 (delay-induced losses).
		\item \textbf{Factual}: Resource tie-up (Annexure 1.1.1) prevented projects (MPRs).
		\item \textbf{Legal}: Section 73, Indian Contract Act, 1872; \textit{Murlidhar Chiranjilal v. Harishchandra Dwarkadas} (1962, Supreme Court) adopted \textit{Hadley v. Baxendale}.
	\end{itemize}
	
	\subsection*{Assumptions}
	\begin{itemize}
		\item 10\% profit per MoRTH; assumes full redeployment potential.
		\item No reputational losses per contract limits.
	\end{itemize}
	
	\subsection*{Narrative}
	Prolonged resource tie-up due to RoW delays (Claim F) prevented WIPL from pursuing other projects, quantified at \texttt{INR 78.61 Crores} via Hudson Formula. Supported by MPRs and deployment data (Annexure 1.1.1), the claim is tied to Clause 13.1 and \textit{Murlidhar v. Harishchandra}, avoiding speculative losses.
	
	% Updated Total
	\section*{Updated Total}
	\begin{itemize}
		\item Primary Claims (A--I): \texttt{INR 451.55 Crores} (Annexure 1, Claim D updated to \texttt{INR 46.23 Crores}).
		\item Supplementary Claims (8--14):
		\[
		\texttt{INR 62 + 25 + 9.15 + 11 + 20 + 35} = \texttt{INR 162.15 Crores}
		\]
		\item Total (S. No. 10):
		\[
		\texttt{INR 451.55 Crores} + \texttt{INR 162.15 Crores} = \texttt{INR 613.62 Crores}
		\]
		\item Interest (S. No. 11):
		\[
		\texttt{INR 613.62 Crores} \times 0.12 \times \frac{86}{365} = \texttt{INR 17.34 Crores}
		\]
		\item Grand Total (S. No. 12):
		\[
		\texttt{INR 613.62 Crores} + \texttt{INR 17.34 Crores} = \texttt{INR 630.96 Crores}
		\]
	\end{itemize}
	
	% Annexure Updates
	\section*{Annexure Updates}
	\begin{itemize}
		\item \textbf{Annexure 1.1.1}: Plant, equipment, labour deployed (month-wise, Schedule 9, Sep 2023). Supports Claims A, B, 8, 9, 13 (182 skilled, 273 unskilled, 12 excavators).
		\item \textbf{Annexure 1.8}: Updated to \texttt{INR 46.23 Crores}, includes cement/steel escalation (Appendix K).
		\item \textbf{Appendices H--N}: MPRs, productivity logs, EOT notices, invoices, BOQ certifications, subcontractor agreements, MoEFCC records.
		\item \textbf{Appendix X}: Delay analysis (as-planned vs. as-built).
	\end{itemize}
	
	% Recommendations
	\section*{Recommendations}
	\begin{itemize}
		\item \textbf{Evidence}: Attach appendices (H--N, X) with logs, notices, certifications. Request Schedule 9, productivity logs, subcontractor agreements, IPC schedules.
		\item \textbf{Interim Payment}: Request \texttt{INR 250 Crores} for undisputed heads (A, B, D, G, 10).
		\item \textbf{Arbitration}: Reserve rights per Clause 23.1; prepare for DRB if Engineer rejects (Clause 20.2).
	\end{itemize}
	
\end{document}